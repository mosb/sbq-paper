\documentclass{article}
\usepackage{preamble}
%\usepackage{subfig}
\newcommand{\dblspace}{\setlength{\baselineskip}{0.8cm}}
\renewcommand{\pskinny}[2]{p\big(#1|#2\big)}
\usepackage{graphicx} % For figures
\usepackage{subfigure} 
\usepackage{natbib}   % For citations
\usepackage{algorithm}
\usepackage{algorithmic}
\usepackage{hyperref}
\newcommand{\theHalgorithm}{\arabic{algorithm}}
\usepackage[normalem]{ulem}  % for strikethrough
\usepackage{color} % for comments to each other
\usepackage{comment}
\usepackage{pgfplots}
% Mike says: the command below caused me compile errors
%\pgfplotsset{compat=newest}
\usepackage{icml2012} 
% \usepackage[accepted]{icml2012}¡

% The \icmltitle you define below is probably too long as a header.
% Therefore, a short form for the running title is supplied here:
\icmltitlerunning{Sampling for Bayesian Quadrature}

\begin{document} 
\twocolumn[
\icmltitle{Sampling for Bayesian Quadrature}

% It is OKAY to include author information, even for blind
% submissions: the style file will automatically remove it for you
% unless you've provided the [accepted] option to the icml2012
% package.
\icmlauthor{Your Name}{email@yourdomain.edu}
\icmladdress{Your Fantastic Institute,
            314159 Pi St., Palo Alto, CA 94306 USA}
\icmlauthor{Your CoAuthor's Name}{email@coauthordomain.edu}
\icmladdress{Their Fantastic Institute,
            27182 Exp St., Toronto, ON M6H 2T1 CANADA}

% You may provide any keywords that you 
% find helpful for describing your paper; these are used to populate 
% the "keywords" metadata in the PDF but will not be shown in the document
\icmlkeywords{Bayesian Quadrature, Monte Carlo, Gaussian Processes, Numerical Integration, Model Evidence, Likelihood ratios}

\vskip 0.3in
]

\begin{abstract} 
%We describe a novel approach to quadrature for probabilistic integrals, offering a competitor to traditional Monte Carlo methods. We use a Bayesian quadrature framework \citep{BZHermiteQuadrature,BZMonteCarlo}.
We introduce several innovations making Bayesian Quadrature methods suitable for computing model evidences, or normalization constants.  We demonstrate the many advantages of model-based integration over standard Markov-chain Monte Carlo approaches.  These include a natural stopping criterion, an estimate of uncertainty in our integral, the ability to use active learning rather than markov chains in order to learn about the function being integrated.
\end{abstract} 

\section{Introduction}

Training or evaluating any probabilistic model typically requires an integration over model parameters, weighted by their likelihoods.  This commmon problem has many names:  computing the model evidence[cite skilling?], estimating the partition function, normalizing a distribution.  Typically, this task is performed using Markov-Chain Monte Carlo (MCMC) methods.  These methods have many well-known problems, (such as requiring extensive tuning, becoming stuck in local modes, and falsely appearing to converge \citep{NealMC}).  However, almost all standard approaches fall into the wide faimly of MCMC methods.

These methods estimate the model evidence given the value of the integrand on a set of sample points, a set that is limited in size by the computational expense of evaluating the integrand. As discussed in \citep{MCUnsound}, traditional Monte Carlo integration techniques do not make the best possible use of this valuable information. An alternative is found in Bayesian quadrature \citep{BZHermiteQuadrature}, that uses function samples within a Gaussian process model to compute a closed-formed posterior over the value of the integral.

\section{Modeling Likelihood Functions}

\begin{align}
& \mean{\inty{\lfn}}{\psi_0,\tvr_s} \nonumber\\
& \deq \int \mean{\psi[\tr]}{\psi_0,\tr}
\p{\tr}{\tvr_s}\, \ud \tr 
\nonumber\\
& = \mean{\psi[\tr]}{\psi_0,m_{\tr|s}} \nonumber\\
& = \mean{\inty{\lfn}}{\vr_s} + \iint \bigl(\mean{r(\lfv)}{\vr_s}+\gamma\bigr)\,\Delta(\lfv)\,\po{\lfv}\ud\lfv
\nonumber\\
& = \mean{\inty{\lfn}}{\vr_s} + \mean{\inty{r \Delta}}{\vr_s} + \gamma\, \mean{\inty{ \Delta}}{\vr_s}
\label{eq:mean_ev}
\end{align}

\bibliography{bub}
\bibliographystyle{icml2012}

\end{document} 